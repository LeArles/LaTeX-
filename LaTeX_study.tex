\documentclass{report}
\usepackage{ctex}
\usepackage{xcolor}       % 用于定义颜色
\usepackage{listings}     % 用于代码排版
\usepackage[textwidth=16cm, textheight=24cm]{geometry}
\usepackage{graphicx}
\usepackage{amsmath}
\usepackage{amssymb}
\usepackage{array}
\usepackage{multirow}
\usepackage{booktabs}
\usepackage{subcaption}
\usepackage{tcolorbox}
\usepackage[ruled, linesnumbered]{algorithm2e}
\usepackage{bussproofs}
\usepackage{bm}
\usepackage{amsthm}

% 定义高亮颜色
\definecolor{keywords}{rgb}{0,0,1}       % 蓝色
\definecolor{comments}{rgb}{0,0.5,0}     % 绿色
\definecolor{strings}{rgb}{1,0,0}        % 红色

\usepackage[
    colorlinks=true,
    linkcolor=pink,
    urlcolor=purple,
    citecolor=green
]{hyperref}
\usepackage{cleveref}

% 定义一个名为 'thm' 的环境,打印为 '定理'
\newtheorem{thm}{定理}[section]
% 定义一个名为 'defn' 的环境,打印为 '定义'
\newtheorem{defn}{定义}
% 2. 定义引理,使用定理的计数器 [thm]
\newtheorem{lem}[thm]{引理}

\title{第一个 \LaTeX 项目——\LaTeX 使用教程}
\author{李弈}
\date{2025年9月29日}

% ------------------- listings 全局设置 -------------------
% 全局设置 listings 样式
\lstset{
    language=[latex]TeX,             % 设置语言为 LaTeX
    basicstyle=\ttfamily,            % 字体为等宽
    keywordstyle=\color{keywords},   % 关键字颜色
    commentstyle=\color{comments},   % 注释颜色
    stringstyle=\color{strings},     % 字符串颜色
    showstringspaces=false,          % 不显示字符串中的空格
    numbers=left,                    % 显示行号,在左侧
    numberstyle=\tiny,               % 行号字体大小
    breaklines=true,                 % 自动换行
    frame=single,                    % 显示一个单边框
}
% --------------------------------------------------------

\begin{document}

\maketitle

\tableofcontents

% 基础篇从这里开始,作为第一个章节(chapter)
\chapter{基础篇}

\section{基本命令与环境}

\subsection{基本命令}
在 \LaTeX 中,\textbf{命令}是用于执行特定操作的指令。它们通常以反斜杠 \verb|\| 开头,后面跟着命令名称。有些命令可以接受参数,这些参数用花括号 \{\} 包裹。常见的命令包括:
\begin{itemize}
    \item \verb|\documentclass{}|:这是每个 \LaTeX 文档的第一个命令,用于声明文档的类型,例如\texttt{article}、\texttt{report} 或 \texttt{book}。
    \item \verb|\usepackage{}|:用于导入宏包,扩展 \LaTeX 的功能。
    \item \verb|\section{}| 和 \verb|\subsection{}|:用于创建文档的标题结构。
    \item \verb|\title{}| 和 \verb|\author{}|:设置文档的标题和作者信息。
    \item \verb|\maketitle|:用于生成并显示标题部分。
    \item \verb|\textbf{}|、\verb|\textit{}|、\verb|\texttt{}|:用于设置文本样式。
    \begin{itemize}
        \item \verb|\textbf{}|:生成\textbf{加粗}的文本。
        \item \verb|\textit{}|:生成\textit{斜体}的文本。
        \item \verb|\texttt{}|:生成\texttt{等宽}的文本,可用于代码和命令。
    \end{itemize}
    \item \verb|\\|:用于手动换行。
\end{itemize}

\subsection{基本环境}
\textbf{环境}用于对文档的某个区域进行特殊的排版,它们由一对 \verb|\begin{}| 和 \verb|\end{}| 命令定义。所有需要特殊处理的内容都放在这对命令之间。常见的环境包括:
\begin{itemize}
    \item \verb|document|:最重要的环境,所有可见的文档必须放在 \verb|\begin{document}| 和\verb|\end{document}| 之间。
    \item \verb|itemsize|:用于创建\textbf{无序列表}。在 \verb|\begin{itemsize}| 和 \verb|\end{itemsize}| 之间,使用 \verb|\item| 命令来标记每一个列表项。具体使用方法如下:
    \begin{lstlisting}
\begin{itemize}
    \item 第一个项目
    \item 第二个项目
\end{itemize}
    \end{lstlisting}
    \item \verb|enumerate|:用于创建\textbf{有序列表}。具体使用方法如下:
        \begin{lstlisting}
\begin{enumerate}
    \item 第一个项目
    \item 第二个项目
\end{enumerate}
        \end{lstlisting}
    \item \verb|center|:用于将文本或内容居中对齐。具体使用方法如下:
        \begin{lstlisting}
\begin{center}
    居中对齐的文本。
\end{center}
        \end{lstlisting}
    \item \verb|figure|:用于管理图片。这是一个“浮动体”环境,\LaTeX 会自动将图片放置在最佳位置,以避免排版不美观。
    \item \verb|table|:用于创建表格。这是一个“浮动体”环境
\end{itemize}

\section{包管理}
在 \LaTeX 中,宏包是一个核心概念。可以把它们理解成扩展 \LaTeX 功能的插件或库。宏包可以做很多基础 \LaTeX 无法直接完成的事情,比如:插入图片、创建复杂的数学公式、改变页面布局,或者使用特定字体等。

要使用一个宏包,需要在文档的\textbf{导言区}中使用 \verb|\usepackage| 命令。导言区是指 \verb|\documentclass{}| 和 \verb|\begin{document}| 之间的部分。

基本语法是:
    \begin{lstlisting}[title=Latex 代码示例]
\usepackage{package_name}
    \end{lstlisting}

有些宏包可以接受\textbf{选项},这些选项用方括号 \verb|[]| 包裹,放在宏包名称的前面。选项用于修改宏包的行为。例如 \verb|geometry| 宏包可以用来设置文档的页边距。代码如下所示:
    \begin{lstlisting}
\usepackage[a4paper, margin=2cm]{geometry}
    \end{lstlisting}

对于初学者来说,以下宏包是必不可少的:
\begin{itemize}
    \item \verb|amsmath|:增强数学公式排版功能。
    \item \verb|graphicx|:在文档中插入图片。
    \item \verb|hyperref|:创建可点击的超链接,让 PDF 文档中的目录,引用和网址都变得可交互。
    \item \verb|xcolor|:提供丰富的颜色选项,为文本、背景上色。
\end{itemize}

\section{章节与段落}
在 \LaTeX 中,需要通过一些规则和命令来告诉 \LaTeX 如何处理文本。

\subsection{章节}
在 \LaTeX 中,可以使用不同的标题命令来创建不同级别的章节和子章节。常用命令为:
\begin{itemize}
    \item \verb|\section{标题}|:一级标题
    \item \verb|\subsection{标题}|:二级标题
    \item \verb|\subsubsection{标题}|:三级标题
\end{itemize}

\subsection{段落}

\LaTeX 将文本组织成段落。默认情况下,它会自动识别段落。要开始一个新的段落,只需要在两个段落之间留一个空行即可。
\begin{itemize}
    \item \textbf{自动缩进}:\LaTeX 会在每个段落的开头自动缩进
    \item \textbf{换行}:无需手动换行,\LaTeX 会根据版面的宽度自动处理
\end{itemize}
示例如下:
\begin{lstlisting}
这是第一个段落

这是第二个段落
\end{lstlisting}

如果想在不开始新段落的情况下\textbf{强制换行},可以使用命令 \verb|\\| 或 \verb|\newline|。

\section{文本}

\subsection{字体大小}
\LaTeX 定义了一系列用于改变字体大小的命令,这些命令通常用于文档的特定部分,比如标题或标注。它们是\textbf{开关}命令,一旦使用就会影响其后面的所有文本,直到遇到另一个字体大小命令或作用范围结束。

因此设置字体大小时,更常用的一个办法如下所示:
\begin{lstlisting}
这是普通大小的文本,{\Large 大的字体} 在这里。
\end{lstlisting}

从大到小排列的字体大小命令:
\begin{itemize}
    \item \verb|\Huge|
    \item \verb|\huge|
    \item \verb|\LARGE|
    \item \verb|\Large|
    \item \verb|\large|
    \item \verb|\normalsize|
    \item \verb|\small|
    \item \verb|\footnotesize|
    \item \verb|\scriptsize|
    \item \verb|\tiny|
\end{itemize}

\subsection{特殊字符}
\LaTeX 对一些字符有特殊的用途,比如 \#、\$、\% 等。如果想在文本中使用这些字符,需要用反斜杠进行转义。
\begin{itemize}
    \item 对于 \# 等普通的特殊字符,写法为:\verb|\#|
    \item 对于 \textbackslash\ 这种特殊字符,写法为:\verb|\textbackslash|
\end{itemize}

\subsection{空格}
几种空格方法如下:
\begin{table}[h!]
    \centering
    \caption{常用空格方法}
    \begin{tabular}{|c|c|c|c|}
        \hline
        空格方式 & 源代码 & 显示 & 宽度 \\ \hline
        普通空格 & \verb|a b| & a b & 一个英文字符的宽度\\ \hline
        quad空格 & \verb|a\quad b| & a\quad b & 一个中文字符的宽度\\ \hline
        qquad空格 & \verb|a\qquad b| & a\qquad b & 两个中文字符的宽度\\ \hline
        大空格 & \verb|a\ b| & a\ b & 1/3 字符的宽度\\ \hline
    \end{tabular}
    \label{tab:sample}
\end{table}

\subsection{换页}
使用 \verb|\newpage| 进行换页,一般在一页的最后写

\section{表格}

\subsection{tabular}

创建表格最常用的环境是 \verb|tabular|。而 \verb|tabular| 环境是用来创建表格内容的,它需要一个参数来定义表格的列格式,基本语法如下:
\begin{lstlisting}[title=Latex 代码示例]
\begin{tabular}{格式}
    \hline %顶部横线,一般来说必须要写
    表头1 & ... & 表头 n \\ [\hline]
    表项11 & ... & 表项1n \\ [\hline]
    ...
    表项n1 & ... & 表项nn \\ [\hline]
\end{tabular}
\end{lstlisting}

\begin{itemize}
    \item \verb|{格式}|:这是一个必要的参数,用于定义表格中每一列的对齐方式。
        \begin{itemize}
            \item \verb|l|:左对齐
            \item \verb|c|:居中对齐
            \item \verb|r|:右对齐
            \item \verb|||:绘制一条垂直线
        \end{itemize}
    \item 对于表头/每一行表项后面的 \verb|[\hline]| 代表 \verb|\hline| 是可以省略的,如果写的话,那么该表头/表项上就会出现一个横线,不写的话就不会出现这条横线
\end{itemize}

对于如下 \LaTeX 代码:
\begin{lstlisting}
\begin{tabular}{|l|c|r|}
\hline % 顶部横线
姓名 & 年龄 & 城市 \\ \hline % 表格头
小明 & 25 & 北京 \\ \hline
小红 & 28 & 上海 \\ \hline
\end{tabular}
\end{lstlisting}

会出现如下效果

\begin{tabular}{|l|c|r|}
\hline % 顶部横线
姓名 & 年龄 & 城市 \\ \hline % 表格头
小明 & 25 & 北京 \\ \hline
小红 & 28 & 上海 \\ \hline
\end{tabular}

\subsection{table}
\verb|tabular| 环境只是创建了表格内容,它不会自动添加标题或编号。如果想让表格拥有标题、编号并且能够像图片一样“浮动”(让 \LaTeX 自动寻找最佳位置放置它),你需要将 \verb|tabular| 环境放在 \verb|table| 浮动体环境中。基本语法如下:
\begin{lstlisting}
\begin{table}[h!]
    \centering
    \caption{一个示例表格}
    \begin{tabular}{|l|c|r|}
        \hline
        表头1 & ... & 表头 n \\ [\hline]
        表项11 & ... & 表项1n \\ [\hline]
        ...
        表项n1 & ... & 表项nn \\ [\hline]
    \end{tabular}
    \label{tab:sample}
\end{table}
\end{lstlisting}

\begin{itemize}
    \item \verb|\caption{}|:为表格添加标题。
    \item \verb|\label{}|:为表格添加表填,之后可以用 \verb|\ref{tab:sample}| 命令来引用这个表格的编号。
    \item \verb|\centering|:将整个表格居中。
    \item \verb|[h!]|:这是一个\textbf{位置指示符},告诉\LaTeX 应该如何放置这个浮动体,常用的位置指示符有:
    \begin{itemize}
        \item h(here):尽可能放在这里
        \item t(top):放在页面的顶部
        \item b(buttom):放在页面的底部
        \item p(page):放在单独的浮动页
    \end{itemize}
\end{itemize}

\section{目录基础}
在 \LaTeX 中创建目录非常简单,而且完全自动化。你不需要手动输入章节名称和页码,\LaTeX 会自动扫描你的文档结构,生成一个完整的、可点击的目录。

创建目录的核心命令只有一个:\verb|\tableofcontents|。你只需要在文档中你希望目录出现的位置(通常在 \verb|\maketitle| 命令之后)插入这个命令即可。

\section{\LaTeX 代码注释}
单行注释:
\begin{lstlisting}[title=Latex 代码示例]
% 注释内容
\end{lstlisting}

多行注释 1:
\begin{lstlisting}[title=Latex 代码示例]
\iffalse
注释内容
\fi
\end{lstlisting}

多行注释 2(使用 \verb|\usepackage{verbatim}|):
\begin{lstlisting}[title=Latex 代码示例]
\begin{comment}
注释内容
\end{comment}
\end{lstlisting}

\section{纸张布局}

\subsection{documentclass}
语法如下:
\begin{lstlisting}[title=Latex 代码示例]
\documentclass[options]{class_name}
\end{lstlisting}
\begin{itemize}
    \item \verb|class_name|:文档类的名称,例如 \verb|article|、\verb|report|或\verb|book|
    \item \verb|options|:可选参数,用于修改文档类的默认配置,可以使用多个选项,不同选项之间用逗号分隔
    \begin{itemize}
        \item 字体大小:设置文档的基础字体大小(例如:\verb|12pt|)
        \item 纸张大小:设置纸张规格(例如:\verb|a4paper|、\verb|b5paper|)
        \item 排版模式:\verb|twocolumn|用于双栏排版,\verb|twoside|用于双面排版
        \item 草稿模式:\verb|draft|会在超出边界的地方用黑线标记,方便检查排版问题
    \end{itemize}
\end{itemize}


\subsection{geometry 宏包}
\verb|geometry| 宏包是 \LaTeX 中最强大和灵活的页面布局工具。它可以让你精确地控制纸张大小、页边距、页眉页脚的位置,从而取代文档类自带的粗略设置。

基本语法如下:
\begin{lstlisting}
\usepackage[选项1,选项2,...]{geometry}
\end{lstlisting}
\begin{itemize}
    \item 纸张大小
    \begin{itemize}
        \item \verb|a4paper|: A4 纸张(默认)。
        \item \verb|letterpaper|: Letter 纸张。
        \item \verb|b5paper|: B5 纸张。
        \item \verb|a3paper|, \verb|a5paper|, \verb|b4paper| 等更多选项。
    \end{itemize}
    \item 页边距
    \begin{itemize}
        \item \verb|margin=2.5cm|: 设置所有四个边距都为 2.5cm。这是最简洁的用法。
        \item \verb|left=2cm|, \verb|right=3cm|: 分别设置左右边距。
        \item \verb|top=2cm|, \verb|bottom=2.5cm|: 分别设置上下边距。
        \item \verb|hmargin=2.5cm|: 设置水平(左右)边距。
        \item \verb|vmargin=3cm|: 设置垂直(上下)边距。
    \end{itemize}
    \item 页眉和页脚
    \begin{itemize}
        \item \verb|headheight=1.2cm|: 设置页眉区域的高度。
        \item \verb|headsep=0.8cm|: 设置页眉底部与正文顶部的距离。
        \item \verb|footskip=1cm|: 设置正文底部与页脚顶部的距离。
    \end{itemize}
    \item 正文区域大小
    \begin{itemize}
        \item \verb|textwidth=15cm|: 设置正文区域的宽度。
        \item \verb|textheight=22cm|: 设置正文区域的高度。
        \item \verb|width=15cm|, \verb|height=22cm|: 与 \texttt{textwidth} 和 \texttt{textheight} 相同。
    \end{itemize}
\end{itemize}

\section{交叉引用}\label{sec:交叉引用}
在 \LaTeX 中,交叉引用是一个非常强大的功能,它能让你在文档中自动引用章节、图表、公式和表格的编号,而无需手动管理这些编号。这不仅省去了很多麻烦,还能确保文档在修改时(例如增删章节)编号依然正确。

\LaTeX 的交叉引用主要依赖于两个命令:\verb|\label| 和 \verb|\ref|

\subsection{基本原理}
\begin{enumerate}
    \item 定义标签(Label):你需要在引用的地方使用 \verb|\label{key}| 来设置一个唯一的标签,这里的 \texttt{key} 是自己定义的。
    \item 引用标签(Reference):在文档的任何其他地方,可使用 \verb|\ref{key}| 来引用这个标签。\LaTeX 编译时会自动查找 \texttt{key} 对应的编号,并将其插入到 \verb|\ref| 的位置。
\end{enumerate}

由于编号的生成和引用是一个两步过程,因此你的 \LaTeX 文档通常需要编译两次才能使所有的引用都正确显示。第一次编译时,\LaTeX 会生成一个 \texttt{.aux} 文件,其中包含了所有的标签及其对应的编号信息;第二次编译时,\LaTeX 会读取这个 \texttt{.aux} 文件,用正确的编号替换 \verb|\ref| 命令。

\subsection{基础使用方法}
可以对以下几种主要对象进行交叉引用:

\textbf{章节}

你可以为任何章节标题设置标签,然后引用它。如下所示:
\begin{lstlisting}
\section{简介}
\label{sec:intro}

这是一个关于章节的例子。正如我们在 \ref{sec:intro} 节中所讨论的,...
\end{lstlisting}

\textbf{图表}

通常在 \texttt{figure} 或 \texttt{table} 浮动环境中设置标签,注意 \verb|\label| 应该放在 \verb|\caption| 命令的\textbf{后面}。如果放在前面,\verb|\ref| 可能会引用到错误的编号(例如章节编号)。具体例子如下所示:
\begin{lstlisting}
\begin{figure}[htbp]
    \centering
    \includegraphics[width=0.7\textwidth]{example-image-a}
    \caption{一个示例图片}
    \label{fig:sample} % 放在这里!
\end{figure}

如图 \ref{fig:sample} 所示,这是一个非常重要的示例。
\end{lstlisting}

\section{URL}\label{sec:URL}
最简单的用法为:引用 \texttt{url} 宏包,之后用 \verb|\url{指定的 URL}| 设置 URL 链接。

例如可以用\verb|\url{https://www.bilibili.com/}| 生成 B 站的 URL:\url{https://www.bilibili.com/}

\texttt{URL} 宏包的特点是:
\begin{itemize}
    \item 自动处理特殊字符
    \item 默认使用等宽字体
    \item 默认不可点击(只显示文本)
\end{itemize}

为了更好地展示 URL 链接(修改颜色、使其可点击等),我们使用 \verb|hyperref| 宏包,该宏包更加全面,具体内容参考:进阶篇——引用进阶。

\section{图片插入基础}

在插入图片之前,需要引入 \verb|graphicx| 宏包。

插入图片的基本命令是:\verb|\includegraphics[参数]{filename}|。这里的 \texttt{filename} 是图片路径名(绝对路径和相对路径都可以)。注意:传统的 \LaTeX 引擎只支持 \texttt{.eps} 格式。而 \texttt{pdfLaTeX}、\texttt{XeLaTeX}、\texttt{LuaLaTeX} 在此基础上还支持 \texttt{.pdf}、\texttt{.jpg}、\texttt{.png} 格式的图片。

\verb|\includegraphics[参数]{filename}| 中的参数用于指定图片的大小。有以下指定方式:
\begin{itemize}
    \item 指定高度和宽度:其中 width 和 height 一般设置为以 cm 为单位
    \begin{lstlisting}
\includegraphics[width=..., height=...]{filename}
    \end{lstlisting}
    \item 指定相对尺寸
    \begin{lstlisting}
% 图片宽度为文本宽度的 a 倍
\includegraphics[width=a\textwidth]{filename}
% 图片高度为文本高度的 a 倍
\includegraphics[height=a\textheight]{filename}
    \end{lstlisting}
    \item 保持宽高比:如果只指定了 \texttt{width} 或 \texttt{height} 中的一个,\LaTeX 会自动保持图片的原始宽高比
\end{itemize}

上面的代码仅能确保图片以一个合适的大小被插入到文档中,如果需要更好地排版文字和图片,通常会把图片放在一个浮动环境中,最常用的是 \texttt{figure} 环境。浮动环境的作用是让 \LaTeX 自动决定图片最佳的放置位置,避免图片将页面分隔得非常突兀。\texttt{figure} 的基本语法如下:
\begin{lstlisting}
\begin{figure}[options]
    \centering
    \includegraphics[图片大小设置选项]{filename}
    \caption{这是图片的标题}
    \label{fig:mylabel}
\end{figure}
\end{lstlisting}

\begin{itemize}
    \item \verb|[options]|:可选参数,用于控制图片在页面上的放置位置。
    \begin{itemize}
        \item \texttt{h}:Here,尽量放在代码所在位置。
        \item \texttt{t}:Top,尽量放在页面的顶部。
        \item \texttt{b}:Bottom,尽量放在页面的底部。
        \item \texttt{p}:Page,单独放在一页上。
        \item \texttt{!}:强制执行,与上述选项结合使用。
    \end{itemize}
    \item \verb|centering|:让图片居中。
    \item \verb|\caption{...}|:为图片添加标题,这个标题会显示在图片下方,并且会出现在图表目录中。
    \item \verb|label{...}|:为图片设置一个标签,可以方便地在正文中进行交叉引用。
\end{itemize}

对于如下代码段,可以如下的插入效果:
\begin{lstlisting}
\begin{figure}[h]
    \centering
    \includegraphics[width=10cm, height=3cm]{江景.jpg}
    \caption{湘江晚霞图}
    \label{fig:xiangjiangwanxia}
\end{figure}

湘江晚霞如图 \ref{fig:xiangjiangwanxia} 所示。
\end{lstlisting}

\begin{figure}[h]
    \centering
    \includegraphics[width=10cm, height=3cm]{江景.jpg}
    \caption{湘江晚霞图}
    \label{fig:xiangjiangwanxia}
\end{figure}

湘江晚霞如图 \ref{fig:xiangjiangwanxia} 所示。

\section{数学公式基础}

\subsection{常用宏包}
为了排版数学公式,最常用的是 \texttt{amsmath} 宏包。当然还有以下常用的宏包:
\begin{itemize}
    \item \texttt{amssymb}:提供了一些额外的数学符号。
    \item \texttt{mathtools}:增强了 \texttt{amsmath} 的功能,提供了更多有用的命令。
    \item \texttt{bm}:用于在数学模式下让符号加粗。
\end{itemize}

\subsection{行内公式和行间公式}
在 \LaTeX 中,数学公式需要放在特定的“数学环境”中。主要有两种类型:行内公式和行间公式。
\begin{itemize}
    \item 行内公式:公式与正文在同一行显示。这适用于简单、短小的表达式。
    \begin{itemize}
        \item 可用 \verb|$...$| 包围公式。
        \item 可用 \verb|\(...\)| 包围公式(更推荐)。
    \end{itemize}
    \item 行间公式:公式独占一行居中显示(如果需要的话会编号)。这适用于重要或复杂的表达式。
    \begin{itemize}
        \item 可用 \verb|$$...$$| 包围公式(一些旧编译器可能不支持)。
        \item 可用 \verb|\[...\]| 包围公式(推荐)。
        \item 可用 equation 环境。
    \end{itemize}
\end{itemize}

\subsection{数学公式排版基础}
以下内容可以直接通过键盘输入:
\begin{itemize}
    \item 英文字母
    \item 数字
    \item \verb|+|、\verb|-|、\verb|=|、\verb|*| 等符号
\end{itemize}

\subsubsection{上下标}
\textbf{上标}:使用\quad \textbf{\^}。例如 \verb|x^2| 表示 \(x^2\)。

\textbf{下标}:使用\textbf{\_}。例如 \verb|x_i| 表示 \(x_i\)。

\subsubsection{分数和根式}
\textbf{分数}:使用 \verb|\frac{分子}{分母}|。例如 \verb|\frac{1}{2}| 表示 \(\frac{1}{2}\)

\textbf{根式}:使用 \verb|\sqrt(expression)|。例如 \verb|\sqrt{b-ac}| 表示 \(\sqrt{b-ac}\)

\subsection{积分、求和和极限}

不定积分公式为 \verb|\int f(x) dx|:
\[ \int f(x) dx \]

而定积分则表示为:\verb|\int_a^b f(x) dx |,其中 a 表示下界,b 表示上界:
\[ \int_a^b f(x) dx \]

对于求和,通常形式为:\verb|\sum_{下标}^{上标} 表达式|。例如 \verb|\[ \sum_{i=1}^n i^2\]| 表示:
\[ \sum_{i=1}^n i^2\]

在行内环境中,求和的上下标会排版在右侧,如 \verb|\( \sum_{i=1}^n i \)| 会表示为:\( \sum_{i=1}^n i \)。若要强制在上下显示,可以在\verb|\sum| 后加上\verb|\limits|。例如 \verb|\( \sum\limits_{i=1}^n i \)| 会表示为:\( \sum\limits_{i=1}^n i \)。

极限公式可以表示为 \verb|\lim_{变量 \to 某个值} 极限式子|,例如 \verb|\lim_{x \to 0} \frac{\sin x}{x}| 可以表示为:
\[ \lim_{x \to 0} \frac{\sin x}{x}\]

在行内,\verb|\( \lim_{x \to 0} f(x) \)| 表示为:\( \lim_{x \to 0} f(x) \),如果要强制在上下显示,可以在\verb|\lim| 后加上\verb|\limits|。

\subsection{矩阵和行列式}
矩阵和行列式的基本格式为:
\begin{lstlisting}
\begin{matrix/pmatrix/bmatrix/Bmatrix/vmatrix/Vmatrix}
a_11 & a_12 & ... & a_1n \\
...
a_n1 & a_n2 & ... & a_nn
\end{matrix/pmatrix/bmatrix/Bmatrix/vmatrix/Vmatrix}
\end{lstlisting}
其中:
\begin{itemize}
    \item \texttt{matrix}:不带任何括号。
    \item \texttt{pmatrix}:圆括号。
    \item \texttt{bmatrix}:方括号。
    \item \texttt{Bmatrix}:大括号。
    \item \texttt{vmatrix}:单竖线,通常用于行列式。
    \item \texttt{Vmatrix}:双竖线,通常用于行列式。
\end{itemize}

例如:
\begin{lstlisting}
\[
\begin{Vmatrix}
1 & 2 & 3 \\
4 & 5 & 6 \\
7 & 8 & 9
\end{Vmatrix}
\]
\end{lstlisting}

表示:
\[
\begin{Vmatrix}
1 & 2 & 3 \\
4 & 5 & 6 \\
7 & 8 & 9
\end{Vmatrix}
\]

\subsection{多行公式}

\subsubsection{align}
\texttt{align} 环境可以对齐多行公式,并为每一行自动编号,如果不需要对公式进行编号,可以使用 \texttt{align*} 环境。此外\verb|\align| 允许通过 \verb|&| 符号指定对齐点。通过插入多个 \verb|&| 符号,你可以创建多列公式,每列独立对齐。
语法为:
\begin{lstlisting}
\begin{align/align*}
公式1 \\
公式2 \\
...
公式n
\end{align/align*}
\end{lstlisting}

例如:
\begin{lstlisting}
\begin{align*}
2x + 3y = 5 \\
4x - y = 3
\end{align*}

\begin{align}
  x &= 1+2 & y &= \sin(\theta) \\
  x' &= 3+4 & y' &= \cos(\phi)
\end{align}
\end{lstlisting}

表示:
\begin{align*}
2x + 3y = 5 \\
4x - y = 3
\end{align*}

\begin{align}
  x &= 1+2 & y &= \sin(\theta) \\
  x' &= 3+4 & y' &= \cos(\phi)
\end{align}

\subsubsection{split}
当单个公式太长需要换行,但你希望它只获得一个编号时,应使用 \texttt{split} 环境。
\begin{itemize}
    \item \texttt{split} 必须嵌套在带编号的公式环境(如 \texttt{equation} 或 \texttt{gather})内部。
    \item 每行公式中最多只有一个对齐点 \verb|&|。
    \item 编号会居中放置在所有行的垂直中心。
\end{itemize}

例如:
\begin{lstlisting}
\begin{equation}
\begin{split}
  V &= \int_D \left( f(x, y) + g(x, y) \right) \, \mathrm{d}A \\
  &= \int_a^b \int_c^d f(x, y) \, \mathrm{d}y \, \mathrm{d}x + \text{剩余项}
\end{split}
\end{equation}
\end{lstlisting}

效果为:
\begin{equation}
\begin{split}
  V &= \int_D \left( f(x, y) + g(x, y) \right) \, \mathrm{d}A \\
  &= \int_a^b \int_c^d f(x, y) \, \mathrm{d}y \, \mathrm{d}x + \text{剩余项}
\end{split}
\end{equation}

\subsubsection{gather}
用于将多行公式居中堆叠,每行公式独立编号。
例如:
\begin{lstlisting}
\begin{gather}
  E = mc^2 \\
  \nabla \cdot \mathbf{D} = \rho \\
  \text{其他公式}
\end{gather}
\end{lstlisting}

效果为:
\begin{gather}
  E = mc^2 \\
  \nabla \cdot \mathbf{D} = \rho \\
  \text{其他公式}
\end{gather}

\subsubsection{multiline}
专为超长公式设计,它自动将第一行左对齐,最后一行右对齐,中间行居中,且只在最后一行编号。

例如:
\begin{lstlisting}
\begin{multline}
  \frac{\partial}{\partial t} \int_{\Omega} \mathbf{D} \cdot \mathbf{B} \, \mathrm{d}V = \iint_{\partial \Omega} (\mathbf{E} \times \mathbf{H}) \cdot \mathrm{d}\mathbf{S} \\
  + \int_{\Omega} \left( \mathbf{J} \cdot \mathbf{E} + \mathbf{M} \cdot \mathbf{H} \right) \, \mathrm{d}V
\end{multline}
\end{lstlisting}

效果为:
\begin{multline}
  \frac{\partial}{\partial t} \int_{\Omega} \mathbf{D} \cdot \mathbf{B} \, \mathrm{d}V = \iint_{\partial \Omega} (\mathbf{E} \times \mathbf{H}) \cdot \mathrm{d}\mathbf{S} \\
  + \int_{\Omega} \left( \mathbf{J} \cdot \mathbf{E} + \mathbf{M} \cdot \mathbf{H} \right) \, \mathrm{d}V
\end{multline}

\subsubsection{cases}
此外,还可以用 cases 环境定义分段函数。语法如下:
\begin{lstlisting}
\begin{cases}
    \text{第一段函数的表达式} & \text{第一段函数适用的条件} \\
    \text{第二段函数的表达式} & \text{第二段函数适用的条件} \\
    ...
    \text{第n段函数的表达式} & \text{第n段函数适用的条件}
\end{cases}
\end{lstlisting}

例如:
\begin{lstlisting}
\[
f(x) =
\begin{cases}
x^2 & \text{if } x \geq 0 \\
-x^2 & \text{if } x < 0
\end{cases}
\]
\end{lstlisting}

表示:
\[
f(x) =
\begin{cases}
x^2 & \text{if } x \geq 0 \\
-x^2 & \text{if } x < 0
\end{cases}
\]

\subsection{公式编号与引用}
使用 \texttt{equation} 环境可以为公式自动编号,并可以为其添加标签以便交叉引用。语法如下
\begin{lstlisting}
\begin{equation}
    公式
    \label{标签}
\end{equation}
\end{lstlisting}

例如:
\begin{lstlisting}
\begin{equation}
    E=mc^2
    \label{eq:einstein}
\end{equation}

质能方程如公式 \eqref{eq:einstein} 所示。
\end{lstlisting}

表示:
\begin{equation}
    E=mc^2
    \label{eq:einstein}
\end{equation}

质能方程如公式 \eqref{eq:einstein} 所示。




% -------------------------------------------------------------------------------

% 进阶篇从这里开始,作为第二个章节(chapter)
\chapter{进阶篇}\label{ch:2}

\section{字体控制进阶}

\subsection{字体颜色}
在 \LaTeX 中控制字体颜色主要依赖于 \texttt{xcolor} 宏包。而 \texttt{xcolor} 提供了几种改变颜色的方法:
\begin{itemize}
    \item \verb|\textcolor{color}{text}|:将 \texttt{text} 的颜色改为 \texttt{color}。例如 \verb|\textcolor{red}{这一句话是红色的}| 的效果为: \textcolor{red}{这一句话是红色的}。
    \item \verb|{\color{color}text}|:将 \texttt{text} 的颜色改为 \texttt{color}。例如 \verb|{\color{blue}这句话是蓝色的}| 的效果为:{\color{blue}这一句话是蓝色的}。
\end{itemize}

前面我们使用的 \texttt{blue}、\texttt{red} 都是 \texttt{xcolor} 的预定义颜色。当然我们可以通过 \verb|\definecolor| 自己定义颜色。具体语法如下:
\begin{lstlisting}
\definecolor{name}{model}{spec}
\end{lstlisting}

\begin{itemize}
    \item \texttt{name}:新颜色的名字。
    \item \texttt{model}:颜色模型,例如 \texttt{rgb}、\texttt{cmyk}、\texttt{HTML} 等。
    \item \texttt{spec}:颜色的数值。
\end{itemize}

例如:
\begin{lstlisting}
\definecolor{mypurple}{RGB}{127, 18, 255}
\textcolor{mypurple}{这是我定义的紫色}
\end{lstlisting}

效果为:

\definecolor{mypurple}{RGB}{127, 18, 255}
\textcolor{mypurple}{这是我定义的紫色}

\subsection{fontspec 使用}
\texttt{fontspec} 提供了几个核心命令来设置文档的字体:
\begin{itemize}
    \item \verb|\setmainfont{font_name}|:设置文档的主字体(通常是衬线字体)。
    \item \verb|\setsansfont{font_name}|:设置无衬线字体。
    \item \verb|\setmonofont{font_name}|:设置等宽字体。
\end{itemize}

有时普通字体、加粗字体、斜体字体可能要求不是同一种字体,此时我们可以为字体添加额外的参数实现上述功能。代码如下(以设置正文字体为例):
\begin{lstlisting}
\setmainfont{Times New Roman}[
    BoldFont = {字体1},
    ItalicFont = {字体2},
    SmallCapsFont = {字体3}
]
\end{lstlisting}

和设置文本颜色类似,可以用 \verb|{\fontspec{font_type} text| 让 \texttt{text} 使用 \texttt{font\_type} 字体。

如果需要同时修改字体和颜色,可以参考以下示例:
\begin{lstlisting}
% 字体为 Arial、颜色为 mypurple
\textcolor{mypurple}{\fontspec{Arial} Arial is purple.}
\end{lstlisting}

\section{引用进阶}

\subsection{hyperref}
\texttt{hyperref} 的核心功能是为 \LaTeX 文档生成可点击的超链接。它可以自动识别文档中的各种引用、目录、URL 和电子邮件地址,并将它们转换为交互式的链接,极大地提升了 PDF 文档的可用性。使用前需要用 \verb|\usepackage[选项]{hyperref}| 加载宏包。一旦加载该宏包,\verb|hyperref| 会自动完成以下设置:
\begin{itemize}
    \item 目录:\verb|\tablecontents| 生成的目录项将变为可点击的链接,可以直接跳转到对应章节。
    \item 交叉引用:所有使用 \verb|\ref|、\verb|\pageref|、\verb|\cite| 和 \verb|\eqref| 的引用都会变为链接。
    \item 页码:\verb|\pageref| 引用的页码也会变为链接。
    \item URL 和邮箱:使用 \verb|\url| 和 \verb|\href| 命令常见的链接都会生效。
\end{itemize}

\verb|\hyperref| 提供了丰富的选项来定制链接的样式和行为,常用选项有:
\begin{itemize}
    \item \texttt{colorlinks=true}:让链接以颜色高亮。
    \item \texttt{linkcolor}:内部链接的颜色,比如章节、图表和公式的引用。
    \item \texttt{filecolor}:指向本地文件的链接颜色。
    \item \texttt{urlcolor}:URL 链接的颜色。
    \item \texttt{citecolor}:引文(bibliography)链接的颜色。
    \item \texttt{pdfpagemode=UseOutlines}:让 PDF 打开时自动显示书签(大纲)。
    \item \texttt{bookmarks=true}:创建 PDF 书签。
\end{itemize}

例如:
\begin{lstlisting}
\usepackage[
    colorlinks=true,
    linkcolor=pink,
    urlcolor=purple,
    citecolor=green
]{hyperref}
\end{lstlisting}

\subsection{cleveref}
它的主要作用是根据引用的类型自动添加描述性文字,从而让文档中的交叉引用更加自然和清晰。使用时需要加载 \texttt{cleveref} 宏包。但是在加载时需要注意:\texttt{cleveref} 必须在 \texttt{hyperref} 之后加载。

\subsubsection{cleveref 基本使用}
\texttt{cleveref} 的核心命令是 \verb|\cref{label}|。它会根据 label 的前缀来判断引用类型,并自动生成正确的引用文本。

常用的标签有:
\begin{itemize}
    \item \texttt{sec}:section
    \item \texttt{fig}:figure
    \item \texttt{tab}:table
    \item \texttt{eq}:equation
    \item \texttt{ch}:chapter
\end{itemize}

例如使用:\verb|\cref{ch:2}| 可以产生一个指向第二章的引用:\cref{ch:2}。

\subsubsection{cleveref 进阶使用}
多重引用:可以一次引用多个标签,\texttt{cleveref} 会自动将它们合并成一个简洁的列表,并处理单数和复数形式,语法如下:
\begin{lstlisting}
\cref{标签1,标前2,...,标签n}
\end{lstlisting}

例如我在文档中定义了两个标签:
\begin{lstlisting}
\label{sec:交叉引用}
\label{sec:URL}
\end{lstlisting}

使用:\verb|\cref{sec:交叉引用,sec:URL}| 可以产生一个 “指向交叉引用” 和 “URL” section 的引用:\cref{sec:交叉引用,sec:URL}

自定义引用格式:可以修改 \texttt{cref} 的默认描述文本,语法如下:
\begin{lstlisting}
\crefname{标签类型}{单数描述}{复数描述}
\end{lstlisting}

例如使用:
\begin{lstlisting}
\crefname{section}{章节}{章节}
\end{lstlisting}

\crefname{section}{章节}{章节}
此时再使用 \verb|\cref{sec:交叉引用,sec:URL}|,产生的效果为:\cref{sec:交叉引用,sec:URL}。

\section{表格插入进阶}

\subsection{multirow 宏包}
在复杂的表格中,经常需要合并行或列的单元格,这主要依赖于 \texttt{multirow} 宏包。使用前需要使用 \verb|\usepackage{multirow}| 进行加载。

\subsubsection{合并列}
\verb|\multicolumn| 命令用于合并列,并允许你重新定义合并后的对齐方式和边界。语法如下:
\begin{lstlisting}
\multicolumn{num}{format}{text}
\end{lstlisting}

\begin{itemize}
    \item \texttt{num}:要合并的列数。
    \item \texttt{format}:合并后的列格式(例如 \texttt{l}、\texttt{c}、\texttt{r} 等)。
    \item \texttt{text}:合并后单元格的内容。
\end{itemize}

例如:
\begin{lstlisting}
\begin{tabular}{|c|c|c|}
\hline
\multicolumn{3}{|c|}{\textbf{学生 A 面试评分}} \\ \hline
90.12 & 89.14 & 93.11 \\ \hline
\end{tabular}
\end{lstlisting}

会产生如下效果:

\begin{tabular}{|c|c|c|}
\hline
\multicolumn{3}{|c|}{\textbf{学生 A 面试评分}} \\ \hline
90.12 & 89.14 & 93.11 \\ \hline
\end{tabular}

\subsubsection{合并行}
\verb|\multirow| 命令用于合并行,语法如下:
\begin{lstlisting}
\multirow{num rows}{width}{text}
\end{lstlisting}

\begin{itemize}
    \item \texttt{num rows}:要合并的行数。
    \item \texttt{width}:单元格的宽度。
        \begin{itemize}
            \item 使用 \texttt{*} 表示自适应宽度。
        \end{itemize}
    \item \texttt{text}:合并后单元格的内容。
\end{itemize}

例如:
\begin{lstlisting}
\begin{tabular}{|c|c|}
\hline
\multirow{2}{*}{\textbf{合并}} & A \\ \cline{2-2}
& B \\ \hline
\end{tabular}
\end{lstlisting}

会产生如下效果:

\begin{tabular}{|c|c|}
\hline
\multirow{2}{*}{\textbf{合并}} & A \\ \cline{2-2}
& B \\ \hline
\end{tabular}

一般来说,\verb|\multirow| 命令会和 \verb|\cline{i-j}| 一起使用。\verb|\cline{i-j}| 的功能为:
\begin{itemize}
    \item 只在指定的列范围(i 到 j)内绘制水平线。
    \item 不会跨越整个表格,适合在合并单元格后局部画线。
\end{itemize}



\subsection{array 宏包}
使用 \verb|\usepackage{array}| 加载 array 宏包后,可以使用以下方法控制换行和对齐方式
\begin{itemize}
    \item \verb|p{width}|:顶部对齐(默认)。
    \item \verb|m{width}|:居中对齐(垂直居中)。
    \item \verb|b{width}|:底部对齐。
    \item 优先级:\texttt{b\{width\}} > \texttt{p\{width\}} > \texttt{m\{width\}}。
\end{itemize}

注意:\texttt{tabular} 环境在对齐表格中的所有行时,遵循一个原则:整行内容是相对于这一行中最高的单元格进行对齐的。因此,如果混用 \texttt{p\{width\}}、\texttt{m\{width\}}、\texttt{b\{width\}},那么可能会造成意想不到的错误。

例如
\begin{lstlisting}
\begin{tabular}{|b{2cm}|m{4cm}|p{6cm}|}
\hline
\textbf{底部对齐} & \textbf{居中对齐} & \textbf{顶部对齐} \\ \hline
这是一个很长的描述,它需要跨越多行,但左边的内容会垂直居中。 
& 这是一个很长的描述,它需要跨越多行,但左边的内容会垂直居中。
& 这是一个很长的描述,它需要跨越多行,但左边的内容会垂直居中。\\ \hline
\end{tabular}
\end{lstlisting}

\begin{tabular}{|b{2cm}|m{4cm}|p{6cm}|}
\hline
\textbf{底部对齐} & \textbf{居中对齐} & \textbf{顶部对齐} \\ \hline
这是一个很长的描述,它需要跨越多行,但左边的内容会垂直居中。 & 这是一个很长的描述,它需要跨越多行,但左边的内容会垂直居中。& 这是一个很长的描述,它需要跨越多行,但左边的内容会垂直居中。\\ \hline
\end{tabular}

\subsection{booktabs 宏包}
基础表格的线条通常很粗且呆板。使用 \texttt{booktabs} 宏包可以创建出具有专业学术风格的、没有垂直线的表格。使用前需要用 \verb|\usepackage{booktabs}| 加载宏包。\texttt{booktabs} 支持的线条类型有:
\begin{itemize}
    \item \verb|\toprule|:表格顶部的粗线。
    \item \verb|\midrule|:表格中部的中等粗细线。
    \item \verb|\bottomrule|:表给底部的粗线。
    \item \verb|\cmidrule[thickness](trim){i-j}|
    \begin{itemize}
        \item \texttt{thickness}:线条粗细(例如:0.5pt)
        \item \texttt{trim}:端点修剪
        \begin{itemize}
            \item \texttt{l}:将线条的左侧稍微缩进一点
            \item \texttt{r}:将线条的右侧稍微缩进一点
        \end{itemize}
        \item \texttt{i-j}(必选):只画第 i 列到第 j 列之间的中等粗细横线
    \end{itemize}
\end{itemize}

例如:
\begin{lstlisting}
\begin{tabular}{lcccc}
\toprule
& \multicolumn{2}{c}{实验组} & \multicolumn{2}{c}{对照组} \\
\cmidrule(r){2-3} \cmidrule(l){4-5}
方法 & 准确率 & 召回率 & 准确率 & 召回率 \\
\midrule
A & 0.92 & 0.88 & 0.85 & 0.80 \\
B & 0.95 & 0.91 & 0.87 & 0.83 \\
\bottomrule
\end{tabular}
\end{lstlisting}

效果如下:

\begin{tabular}{lcccc}
\toprule
& \multicolumn{2}{c}{实验组} & \multicolumn{2}{c}{对照组} \\
\cmidrule(r){2-3} \cmidrule(l){4-5}
方法 & 准确率 & 召回率 & 准确率 & 召回率 \\
\midrule
A & 0.92 & 0.88 & 0.85 & 0.80 \\
B & 0.95 & 0.91 & 0.87 & 0.83 \\
\bottomrule
\end{tabular}

\section{图片插入进阶}

\subsection{并排显示图片}
如果需要将多张图片放在一个 \texttt{figure} 环境里面,并且每张图片都有自己的小标题,此时需要用到 \texttt{subcaption} 宏包。故需要用 \verb|\usepackage{subcaption}| 引入宏包。

语法如下:
\begin{lstlisting}
\begin{figure}[位置参数]
    \centering
    \begin{subfigure}[对齐方式]{宽度}
        \centering
        \includegraphics[子图宽度]{图片文件}
        \caption{子图标题}
        \label{子图标签}
    \end{subfigure}
    \hfill % 或 \quad, \qquad, \hspace{...} 添加水平间距
    \begin{subfigure}[对齐方式]{宽度}
        ...
    \end{subfigure}
    \caption{主图标题}
    \label{主图标签}
\end{figure}
\end{lstlisting}

\begin{itemize}
    \item 位置参数:
    \begin{itemize}
        \item \texttt{h}:here
        \item \texttt{t}:top
        \item \texttt{b}:bottom
        \item \texttt{p}:page
    \end{itemize}
    \item 对齐方式:
    \begin{itemize}
        \item \texttt{t}:顶部对齐
        \item \texttt{b}:底部对齐
        \item \texttt{c}:居中对齐
    \end{itemize}
    \item 宽度:指定子图环境的宽度
    \item \verb|\hfill|:在子图之间添加弹性水平间距,使它们分别靠左和靠右
    \item 子图宽度:一般设置为:\verb|width=linewidth|
\end{itemize}

例如:
\begin{lstlisting}
\begin{figure}[h]
    \centering
    \begin{subfigure}[b]{0.4\textwidth}
        \centering
        \includegraphics[width=\textwidth]{统计图1.png}
        \caption{10000次掷骰子结果统计图}
        \label{fig:touzi1}
    \end{subfigure}
    \hfill % 在两个 subfigure 之间添加水平空间
        \begin{subfigure}[b]{0.4\textwidth}
        \centering
        \includegraphics[width=\textwidth]{统计图2.png}
        \caption{100000次掷骰子结果统计图}
        \label{fig:touzi2}
    \end{subfigure}
    \caption{掷骰子实验结果统计}
    \label{fig:main}
\end{figure}
\end{lstlisting}

效果为:

\begin{figure}[h]
    \centering
    \begin{subfigure}[b]{0.4\textwidth}
        \centering
        \includegraphics[width=\textwidth]{统计图1.png}
        \caption{10000次掷骰子结果统计图}
        \label{fig:touzi1}
    \end{subfigure}
    \hfill % 在两个 subfigure 之间添加水平空间
        \begin{subfigure}[b]{0.4\textwidth}
        \centering
        \includegraphics[width=\textwidth]{统计图2.png}
        \caption{100000次掷骰子结果统计图}
        \label{fig:touzi2}
    \end{subfigure}
    \caption{掷骰子实验结果统计}
    \label{fig:main}
\end{figure}

\subsection{图片跨双栏}
如果文档是双栏排版的,但某张图片需要占据整个页面宽度,则必须使用带星号的 \texttt{figure*} 环境。它通常会放在页面的顶部或单独一页。语法如下:
\begin{lstlisting}
\begin{figure*}
    \centering
    \includegraphics[图片大小]{图片路径}
    \caption{图片标题}
    \label{标签}
\end{figure*}
\end{lstlisting}
 
\section{tcolorbox}
\texttt{tcolorbox} 是 \LaTeX 中一个功能极其强大和灵活的工具,专门用于创建带有背景色、边框、标题和阴影的彩色盒子。它极大地简化了美观提示、定义、定理、代码框等复杂环境的创建。

\subsection{创建单个盒子}
创建单个盒子的语法为:
\begin{lstlisting}
\begin{tcolorbox}[选项]
文本内容
\end{tcolorbox}
\end{lstlisting}

常用选项如下:
\begin{itemize}
    \item 颜色:
    \begin{itemize}
        \item \texttt{colback}:背景色。
        \item \texttt{colframe}:边框色。
        \item \texttt{coltitle}:标题颜色。
    \end{itemize}
    \item 边框:
    \begin{itemize}
        \item \texttt{boxrule}:边框粗细。
        \item \texttt{sharp corners}:使用直角(默认圆角)。
        \item \texttt{arc}:边框圆角的半径。
    \end{itemize}
    \item 标题:
    \begin{itemize}
        \item \texttt{title}:标题文本
        \item \texttt{fonttitle}:标题字体样式
    \end{itemize}
    \item 尺寸:
    \begin{itemize}
        \item \texttt{width}:强制设置盒子固定宽度
        \item \texttt{auto adjust textwidth=true}:让盒子自动适应文本宽度
    \end{itemize}
\end{itemize}

例如:
\begin{lstlisting}
\begin{tcolorbox}[
    colback=red!5!white,          % 背景色:5% 红色 + 95% 白色 (淡红)
    colframe=red!75!black,        % 边框色:75% 红色 + 25% 黑色 (深红)
    title=\textbf{注意 (Attention)},    % 盒子标题,自动居中
    boxrule=1.5pt,                % 边框粗细
    sharp corners,                % 边框使用直角 (默认是圆角)
]
    这个盒子用来强调重要的信息。
\end{tcolorbox}
\end{lstlisting}

效果为:

\begin{tcolorbox}[
    colback=red!5!white,          % 背景色:5% 红色 + 95% 白色 (淡红)
    colframe=red!75!black,        % 边框色:75% 红色 + 25% 黑色 (深红)
    title=\textbf{注意 (Attention)},    % 盒子标题,自动居中
    boxrule=1.5pt,                % 边框粗细
    sharp corners,                % 边框使用直角 (默认是圆角)
]
    这个盒子用来强调重要的信息。
\end{tcolorbox}

\subsection{创建自定义可重复盒子}
在文档中,通常需要多次使用相同样式的盒子。最佳实践是使用 \verb|\newtcolorbox| 在导言区定义一个盒子。语法为:
\begin{lstlisting}
% 现阶段先不考虑参数的用法
\newtcolorbox{盒子名称}[参数的数量][默认参数]{选项}
\end{lstlisting}

使用语法为:
\begin{lstlisting}
\begin{盒子名称}
文本内容
\end{盒子名称}
\end{lstlisting}

\section{算法表示}
为了在 \LaTeX 文档中清晰地表示算法,通常使用 \texttt{algorithm2e}。

使用前需要在导言区用 \verb|\usepackage[ruled, linesnumbered]{algorithm2e}| 加载宏包。
\begin{itemize}
    \item \texttt{ruled}:为算法添加顶部和底部的横线。
    \item \texttt{linesnumbered}:为每一行代码添加行号。
\end{itemize}

之后就可以用 \texttt{algorithm} 环境来编写算法。语法如下:
\begin{lstlisting}
\begin{algorithm}
    \caption{算法标题}
    \label{标签}
    \KwData{输入}
    \KwResult{输出}

    算法伪代码
\end{algorithm}
\end{lstlisting}

算法伪代码中常用命令如下:
\begin{itemize}
    \item \verb|\BlankLine|:插入一个空白行,用于视觉分组。
    \item \verb|\tcp{...}|:注释。
    \item \verb|\KwSty{...}|:手动设置粗体关键字。
    \item \verb|\KwTo|:for 循环中表示 “到”。
    \item \verb|\If{condition}{if-statement}\Else{else-statement}|:条件判断。
    \item \verb|\While{condition}{while-statement}|:while 循环。
    \item \verb|\For{init; cond; step}{for-statement}|:for 循环。
    \item \verb|\Foreach{item \KwIn list}{for-each-statement}|:for-each 循环。
    \item \verb|Repeat{statement}\Until{condition}|:repeat 循环。
    \item \verb|\Function{Name}{Params}{function-body}|:函数。
    \item \verb|\Return{...}|:返回值。
\end{itemize}

例如:
\begin{lstlisting}
\begin{algorithm}[h!]
    \SetAlgoLined % 结构块带线
    \caption{迭代斐波那契数列}
    \label{alg:fibonacci_iter}

    \KwIn{非负整数 $n$}
    \KwOut{斐波那契数列的第 $n$ 项 $F_n$}

    \If{$n \leq 1$}{
        \Return{$n$}\;
    }

    $a \leftarrow 0$\;
    $b \leftarrow 1$\;
    $i \leftarrow 2$\;

    \While{$i \leq n$}{
        $temp \leftarrow a + b$\;
        $a \leftarrow b$\;
        $b \leftarrow temp$\;
        $i \leftarrow i + 1$\;
    }
    
    \Return{$b$}\;

\end{algorithm}
\end{lstlisting}

产生如下效果:

\begin{algorithm}[h!]
    \SetAlgoLined % 结构块带线
    \caption{迭代斐波那契数列}
    \label{alg:fibonacci_iter}
    \KwIn{非负整数 $n$}
    \KwOut{斐波那契数列的第 $n$ 项 $F_n$}
    \If{$n \leq 1$}{
        \Return{$n$}\;
    }
    $a \leftarrow 0$\;
    $b \leftarrow 1$\;
    $i \leftarrow 2$\;
    \While{$i \leq n$}{
        $temp \leftarrow a + b$\;
        $a \leftarrow b$\;
        $b \leftarrow temp$\;
        $i \leftarrow i + 1$\;
    }
    \Return{$b$}\;
\end{algorithm}

\section{代码块}
可以使用 \texttt{listings} 宏包来插入代码块。

使用前需要在导言区用 \verb|\usepackage{listings}| 加载宏包。

\subsection{插入简单代码}
对于行内代码,可以用 \verb|\lstinline{代码}| 完成插入。例如 \verb|\lstinline{int x = 1;}| 的效果为:\lstinline{int x = 1;}

更常用的是独立代码块,语法如下:\verb|\begin{lstlisting} 代码 \end{lstlisting}|。此处的代码可以换行。

\subsection{全局设置}
有时我们会重复用到一个代码块样式,此时可以通过 \verb|\lstset{}| 全局设置样式。常见设置如下:
\begin{itemize}
    \item \texttt{basicstyle}:基础字体款式(例如:\verb|\ttfamily\small|)
    \item \texttt{keywordstyle}:关键字颜色
    \item \texttt{commmentstyle}:注释样式
    \item \texttt{stringstyle}:字符串颜色
    \item \texttt{numbers}:行号的位置(left 或 right)
    \item \texttt{numberstyle}:行号样式
    \item \texttt{frame}:添加边框
    \item \texttt{tabsize}:Tab 等于多少个空格
    \item \texttt{breaklines}:自动换行
    \item \texttt{language}:默认编程语言
    \item \texttt{captionpos}:标题位置
\end{itemize}

例如:
\begin{lstlisting}
\lstset{
    basicstyle=\ttfamily\small,
    keywordstyle=\color{blue},
    commentstyle=\color{gray}\itshape,
    stringstyle=\color{red},
    numbers=left,
    numberstyle=\tiny\color{gray},
    frame=single,
    tabsize=4,
    breaklines=true,
    breakatwhitespace=false,
    showspaces=false,
    showstringspaces=false,
    language=Python,
    captionpos=b,
    escapeinside={(*@}{@*)},
}
\end{lstlisting}

\subsection{从外部文件插入代码}
语法如下:
\begin{lstlisting}
\lstinputlisting[language=编程语言, caption={标题}]{代码源文件位置}
\end{lstlisting}

\section{数学公式进阶}

\subsection{数理逻辑}
大部分逻辑符号都可以通过基本的 \LaTeX 数学环境或结合 \texttt{amssymb}、\texttt{amsmath} 来实现。使用前需要导入宏包。常见的符号有:
\begin{itemize}
    \item \verb|\neg|:$\neg$
    \item \verb|\land|:$\land$
    \item \verb|\lor|:$\lor$
    \item \verb|\oplus|:$\oplus$
    \item \verb|\to|:$\to$
    \item \verb|\implies|:$\implies$
    \item \verb|\leftrightarrow|:$\leftrightarrow$
    \item \verb|\iff|:$\iff$
    \item \verb|\forall|:$\forall$
    \item \verb|\exists|:$\exists$
    \item \verb|\exists!|:$\exists!$
    \item \verb|\nexists|:$\nexists$
    \item \verb|\vdash|:$\vdash$
    \item \verb|\models|:$\models$
    \item \verb|\because|:$\because$
    \item \verb|\therefore|:$\therefore$
\end{itemize}

\subsection{导数和微分}

\subsubsection{基础的微分符号}
\begin{itemize}
    \item \verb|\mathrm{d}x|:$\mathrm{d}x$
    \item \verb|\partial x|:$\partial x$
\end{itemize}

\subsubsection{导数的莱布尼茨记法}
\begin{itemize}
    \item \verb|$\frac{\mathrm{d} y}{\mathrm{d} x}$|:$\frac{\mathrm{d} y}{\mathrm{d} x}$
    \item \verb|$\frac{\mathrm{d}^n y}{\mathrm{d} x^n}$|:$\frac{\mathrm{d}^n y}{\mathrm{d} x^n}$
    \item \verb|\frac{\partial f}{\partial x}|:$\frac{\partial f}{\partial x}$
    \item \verb|\frac{\partial^n f}{\partial x^n}|:$\frac{\partial f}{\partial x}$
    \item \verb|\frac{\partial^2 f}{\partial x \partial y}|:$\frac{\partial^2 f}{\partial x \partial y}$
\end{itemize}

对于导数在特定点的值,可以使用定界符 \verb|\left.| 和 \verb+\right|_+。
\begin{itemize}
    \item \verb|\left.|:是一个空的左定界符,确保右侧的 \texttt{|} 能够正确垂直拉伸。
    \item \verb+\right|_+:会生成一个垂直线,并在其底部添加下标。
\end{itemize}

例如:
\begin{lstlisting}
\left.\frac{\mathrm{d}f}{\mathrm{d}x}\right|_{x=a} = 2a
\end{lstlisting}

表示:
$$\left.\frac{\mathrm{d}f}{\mathrm{d}x}\right|_{x=a} = 2a$$

\subsubsection{导数的拉格朗日记法}
\begin{itemize}
    \item \verb|f'(x)|:$f'(x)$
    \item \verb|f''(x)|:$f''(x)$
    \item \verb|f^{(n)}(x)|:$f^{(n)}(x)$
\end{itemize}

对于导数在 $x_0$ 的值,直接将 $x$ 替换为 $x_0$ 即可。

\subsection{自定义符号}
最基本的自定义命令的语法如下:
\begin{lstlisting}
\newcommand{命令名称}[参数个数][参数默认值]{被替换的代码(可能包含被替换的内容)}
\end{lstlisting}

\begin{itemize}
    \item 参数在被替换的代码中以 \verb|#1|,\verb|#2|,... 的形式引用。
    \item 总参数个数:包含可选参数在内的所有参数数量。
    \item 默认值:可选参数的默认值。
\end{itemize}

\newcommand{\R}{$\mathbb{R}$}
\newcommand{\set}[1]{\left\{ #1 \right\}}

例如:
\begin{itemize}
    \item \verb|\newcommand{\R}{$\mathbb{R}$}|:此后可以用 \verb|\R| 表示 \R。
    \item \verb|\newcommand{\set}[1]{\left\{ #1 \right\}}|:此后可以用 \verb|\set{集合元素}| 表示集合。例如:\verb|\set{1, 2, 3, 4, 5}| 表示 $\set{1, 2, 3, 4, 5}$。
\end{itemize}

如果我们定义了一个已经存在的命令,就会出现错误,为避免该错误,可用 \verb|\providecommand|。它只会在命令尚未定义时创建它,否则不做任何操作。语法如下:
\begin{lstlisting}
\providecommand{命令名称}[参数个数]{替换内容}
\end{lstlisting}

有时我们需要修改一个已存在的命令(例如修改 \verb|\chapter|、\verb|\section| 的外观,或者替换一个简单的符号),此时需要使用 \verb|\renewcommand|,语法如下。
\begin{lstlisting}
\renewcommand{已存在命令名称}[参数个数]{新替换内容}
\end{lstlisting}

例如可以使用下面命令将无序列表符号从圆形改为方块:
\begin{lstlisting}
\renewcommand{\labelitemi}{\ensuremath{\blacksquare}}
\end{lstlisting}

\subsection{公式中的文本}
最常用且推荐的在数学公式中插入文本的命令是 \verb|\text{...}|,使用前需要加载 \verb|amsmath| 宏包。它有如下特点:
\begin{itemize}
    \item 字体切换:\verb|\text{...}| 会暂时跳出数学模式,切换回当前环境下的文本字体。这意味着它会保持正常的文本样式、粗细和大小。
    \item 保留空格:\verb|\text{...}| 内部的空格会被保留。
    \item 自动调整大小:在上下标或分数中,\verb|\text| 内部的字体大小会根据上下文自动缩放。
\end{itemize}

例如:
\begin{lstlisting}
$$
\sum_{i=1}^n \left( \frac{1}{i^2} \right) \quad \text{和} \quad \sum_{i=1}^{n \text{ 偶数}} \left( \frac{1}{i^2} \right)
$$
\end{lstlisting}

效果为:
$$
\sum_{i=1}^n \left( \frac{1}{i^2} \right) \quad \text{和} \quad \sum_{i=1}^{n \text{ 偶数}} \left( \frac{1}{i^2} \right)
$$

\subsection{符号和字体的精细控制}

\subsubsection{常规数学字体}
在数学模式中,变量默认是斜体,而函数名和单位默认是正体。为了更好地排版,\LaTeX 提供了一系列命令来手动切换数学公式中字母和数字的字体样式。
\begin{itemize}
    \item \verb|\mathrm{...}|(罗马正体):用于微分符号、单位、或自定义函数名。例如 \verb|\mathrm{d}x| 的效果为:$\mathrm{d}x$a。
    \item \verb|\mathbf{...}|(数字粗体):用于粗体向量或矩阵(仅对英文字母有效)。例如 \verb|\mathbf{v}| 的效果为:$\mathbf{v}$。
    \item \verb|\mathsf{...}|(无衬线):例如 \verb|\mathsf{E=mc^2}| 的效果为:$\mathsf{E=mc^2}$。
    \item \verb|\mathtt{...}|(等宽打印机字体):例如 \verb|\mathtt{Code}| 的效果为:
    $\mathtt{Code}$。
\end{itemize}

\subsubsection{特殊数学字体}
\begin{itemize}
    \item \verb|\mathcal{...}|:常用于集合、算子等(仅限于大写字母)。例如 \verb|\mathcal{T}| 的效果为:$\mathcal{T}$。
    \item \verb|\mathbb{...}|:常用于表示数集,如实数、整数等(仅限于大写字母)。例如 \verb|\mathbb{R}| 的效果为:$\mathbb{R}$。
    \item \verb|\mathfrak{...}|:常用于李代数等。例如 \verb|\mathfrak{A}| 的效果为:$\mathfrak{A}$。
\end{itemize}

\subsubsection{希腊字母和符号的粗体控制}
\verb|\mathbf{...}| 无法使希腊字母和大多数数学符号变粗。此时可以使用 \texttt{bm} 宏包。注意,在加载 \texttt{bm} 宏包时,确保 \texttt{bm} 宏包在 \texttt{amsmath} 宏包之后加载(如果同时使用)。语法如下:
\begin{lstlisting}
\bm{需要加粗的内容}
\end{lstlisting}

例如 \verb|$\bm{\sigma}$| 的效果为:$\bm{\sigma}$

\subsection{定理类环境与结构化数学写作}
在 \LaTeX 中,定理类环境是结构化数学写作的基石。它们用于清晰地展示定理、引理、定义、命题等关键概念,并实现自动编号和统一的格式。

这个功能主要由 \texttt{amsmath} 宏包的配套宏包 \texttt{amsthm} 实现。因此使用前需要先引入宏包 \texttt{amsthm}。

\subsubsection{newtheorem}
所有定理类环境都必须在文档的导言区使用 、\verb|\newtheorem| 命令进行定义。基本语法如下:
\begin{lstlisting}
\newtheorem{thm/defn/prop/lem}[共享计数器对象]{打印名称}[编号方式]
\end{lstlisting}

\begin{itemize}
    \item 共享计数器:和共享计数器对象共享一个计数器。
    \item 打印名称:实际显示在文档中的文字,例如 “定理”、“定义”、“引理”等。
    \item 编号方式:将定理编号与文档结构(例如 section、chapter 等)挂钩,并在每章/节开始时重置计数器。
\end{itemize}

例如:
\begin{lstlisting}
% 定义一个名为 'thm' 的环境,打印为 '定理'
\newtheorem{thm}{定理}[section]
% 定义一个名为 'defn' 的环境,打印为 '定义'
\newtheorem{defn}{定义}
% 2. 定义引理,使用定理的计数器 [thm]
\newtheorem{lem}[thm]{引理}
\end{lstlisting}

一旦定义,就可以在文档主体中使用,使用方法如下:
\begin{lstlisting}
\begin{thm/defn/prop/lem}[该thm/defn/prop/lem的名字]
具体内容
\end{thm/defn/prop/lem}
\end{lstlisting}

例如:
\begin{lstlisting}
\begin{thm}[连续和可积的关系]
如果函数 $f$ 在区间 $[a, b]$ 上连续,那么它在该区间上必可积。
\end{thm}

\begin{defn}
称函数 $f(x)$ 是连续的,如果对任意 $\epsilon > 0$,存在 $\delta > 0$,使得当 $|x - x_0| < \delta$ 时,有 $|f(x) - f(x_0)| < \epsilon$。
\end{defn}

\begin{lem} % 结果:引理 2.8.2
这是一条引理
\end{lem}   
\end{lstlisting}

表示:
\begin{thm}[连续和可积的关系]
 如果函数 $f$ 在区间 $[a, b]$ 上连续,那么它在该区间上必可积。
\end{thm}

\begin{defn}
称函数 $f(x)$ 是连续的,如果对任意 $\epsilon > 0$,存在 $\delta > 0$,使得当 $|x - x_0| < \delta$ 时,有 $|f(x) - f(x_0)| < \epsilon$。
\end{defn}

\begin{lem} % 结果:引理 2.8.2
这是一条引理
\end{lem}

\subsubsection{样式控制}
\texttt{amsthm} 宏包提供了 \verb|\theoremstyle| 命令来修改后续 \verb|\newtheorem| 环境的默认外观。它必须在 \verb|\newtheorem| 之前使用。使用方法如下:
\begin{lstlisting}
\theorem{plain/definition/remark}
\newtheorem...
\end{lstlisting}

\begin{itemize}
    \item \texttt{plain}(默认样式):标题粗体,正文斜体。主要用于定理、引理、命题、推论。
    \item \texttt{definition}:标题粗体,正文正体。主要用于定义、例题、性质、公理。
    \item \texttt{remark}:标题斜体、正文正体。主要用于注释、评注、例子。
\end{itemize}

\subsubsection{证明环境 proof}
\texttt{amsthm} 宏包提供了一个专门的 \texttt{proof} 环境,用于排版证明文本。使用方法如下:
\begin{lstlisting}
\begin{proof}[证明标题]
证明内容
\end{proof}
\end{lstlisting}

它会自动在开头打印 "证明"(或 "Proof."),并在末尾添加一个证毕符号(Q.E.D. 符号,通常是一个实心小方块 ■)。

例如:
\begin{lstlisting}
\begin{proof}
    证明方式如下...
\end{proof}
\end{lstlisting}

效果为:
\begin{proof}
    证明方式如下...
\end{proof}

如果证明的最后一行是一个行间公式环境(如 \texttt{align*} 或 \texttt{equation*}),证毕符号默认会出现在下一行。为了让它紧跟在公式后面,需要在公式环境内使用 \verb|\qedhere|。

例如:
\begin{lstlisting}
\begin{proof}
  ...所以我们得到:
  \begin{equation*}
    a^2 + b^2 = c^2 \qedhere
  \end{equation*}
\end{proof}
\end{lstlisting}

效果为:
\begin{proof}
  ...所以我们得到:
  \begin{equation*}
    a^2 + b^2 = c^2 \qedhere
  \end{equation*}
\end{proof}

\section{文献管理与引用——BibTeX}

\subsection{BibTeX 简单介绍}
BibTeX 是 \LaTeX 生态系统中处理参考文献和引用的经典且强大的工具。极大地简化了学术写作中的引用管理工作。整个 BibTeX 引用系统涉及三个文件。
\begin{itemize}
    \item 数据库文件:包含所有引文信息(作者、标题、年份)的纯文本文件。
    \item 样式文件:定义参考文献列表和正文引用格式的文件。
    \item 主文档文件:\LaTeX 源代码文件,包含正文和引用命令。
\end{itemize}

\subsection{编译流程}
编译流程为:\texttt{pdflatex} $\xrightarrow{}$ \texttt{bibtex} $\xrightarrow{}$ \texttt{pdflatex} $\xrightarrow{}$ \texttt{pdflatex}

\subsection{.bib 数据库文件}
\verb|.bib| 文件是一个纯文本文件,存储您的所有文献条目。

每个条目都以 \verb|@| 符号开头,包含 3 部分:
\begin{itemize}
    \item 条目类型:
    \begin{itemize}
        \item \verb|@article|:期刊文章。
        \item \verb|@book|:整本书。
        \item \verb|@inproceedings|:会议论文。
        \item \verb|@incollection|:书籍中的章节/论文。
        \item \verb|@phdthesis|:学位论文。
        \item \verb|@misc|:无法归类的文献。
    \end{itemize}
    \item 引用键:花括号中的第一个参数。这是在 \verb|.tex| 文件中用来引用的唯一标识符(\verb|cite| 命令会用到)。
    \item 字段:作者、标题、年份等信息。用逗号分隔。
\end{itemize}

多个作者使用 \verb|AND| 连接。

例如:
\begin{lstlisting}
@article{
  feynman1965quantum,
  author    = {Feynman, Richard P. and Vernon, Frank L. and Hellwarth, Robert W.},
  title     = {Geometrical representation of the Schrödinger equation for a two-level system},
  journal   = {Journal of Applied Physics},
  volume    = {36},
  number    = {11},
  pages     = {3560--3562},
  year      = {1965},
  publisher = {AIP Publishing}
}
\end{lstlisting}

有关 \verb|.bib| 条目的信息了解即可。

\subsection{.bst 样式文件}
\verb|.bst| 样式文件定义了引文和参考文献列表的排版规则。它们决定了:
\begin{itemize}
    \item 文献列表的排序方式(按作者字母、引用顺序等)。
    \item 正文引用的格式(数字、作者-年份)。
    \item 参考文献条目的具体格式(粗体、斜体、逗号、句号)。
\end{itemize}

常见的样式有:
\begin{itemize}
    \item \verb|plain|:
    \begin{itemize}
        \item 按作者字母顺序排序,正文引用格式为数字。
        \item 标准、简介的科技论文格式。
    \end{itemize}
    \item \verb|unsrt|:
    \begin{itemize}
        \item 按引用出现的顺序排序,正文引用格式为数字。
        \item 适合实验报告或按重要性排序的文档。
    \end{itemize}
    \item \verb|abbrv|:
    \begin{itemize}
        \item 按作者字母顺序排序,正文引用格式为数字。
        \item 类似\verb|plain|,但缩写了期刊名和月份。
    \end{itemize}
    \item \verb|alpha|:
    \begin{itemize}
        \item 按作者字母顺序排序,正文引用格式为字母/数字组合。
        \item 使用作者姓氏和年份的组合作为标签。
    \end{itemize}    
\end{itemize}

\subsection{.tex 中实现引用}
使用 \verb|cite| 命令将引文插入到正文中。语法如下:
\begin{lstlisting}
\cite{文献的引用键}
\end{lstlisting}

通常需要在文档末尾生成参考文献列表。为实现这一功能,需要在文档末尾(\verb|\appendix| 或 \verb|\end{document}| 之前),添加以下两个核心命令:
\begin{enumerate}
    \item \verb|\bibliographystyle{样式名}|:指定用于格式化的 \verb|.bst| 文件。
    \begin{itemize}
        \item 例如:\verb|\bibliographystyle{plain}|。
    \end{itemize}
    \item \verb|\bibliography{bib文件名}|:指定你的 \verb|.bib| 文件名(不包含 \verb|.bib| 后缀)。
    \begin{itemize}
        \item 如果有一个数据库文件名为 \verb|myref.bib|,则使用 \verb|\bibliography{myref}|。
    \end{itemize}
\end{enumerate}

一个例子如下:
\begin{lstlisting}
\begin{document}
  % ... 正文内容 ...
  
  \nocite{*} % 可选:将 .bib 文件中的所有文献都列出,即使没有在正文中引用。
  
  \newpage
  \bibliographystyle{plain}
  \bibliography{myreferences}

\end{document}
\end{lstlisting}

另一个例子:
\begin{lstlisting}
参考这篇论文 \cite{10.5555/3295222.3295349}
\end{lstlisting}

效果为:参考这篇论文 \cite{10.5555/3295222.3295349}

\subsection{natbib 宏包}
虽然 BibTeX 本身功能强大,但其原生的 \verb|\cite| 命令比较单一。\verb|natbib| 宏包(需在导言区加载 \verb|\usepackage{natbib}|)是增强经典 \verb|BibTeX| 引用的标准工具,它允许更灵活的引用格式:
\begin{itemize}
    \item \verb|\citep{key}|:括号引用。
    \begin{itemize}
        \item 默认数字样式输出:\verb|[1]|
        \item 默认作者-年份样式输出:\verb|(Einstein, 1905)|
    \end{itemize}
    \item \verb|\citet{key}|:作者引用。
    \begin{itemize}
        \item 默认数字样式输出:\verb|Einstein [1]|
        \item 默认作者-年份样式输出:\verb|Einstein(1905)|
    \end{itemize}
    \item \verb|\citeauthor{key}|:仅作者名。
    \begin{itemize}
        \item 默认数字样式输出:\verb|Einstein|
        \item 默认作者-年份样式输出:\verb|Einstein|
    \end{itemize}
    \item \verb|\citeyear{key}|:仅年份。
    \begin{itemize}
        \item 默认数字样式输出:\verb|1905|
        \item 默认作者-年份样式输出:\verb|1905|
    \end{itemize}
\end{itemize}

\newpage
\bibliographystyle{plain}
\bibliography{exampleref}

\end{document}